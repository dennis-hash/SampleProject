\subsection{SWVL}
In November 2020, SWVL launched a long distance ride sharing services by partnering with Matatu operators. The service was launched  in 12 routes, connecting Naivasha, Nakuru, Molo, Eldoret, Narok, Bomet, Kericho, Kisii, Kisumu, Nyeri, Nanyuki and Machakos, \citep{alvin2020}. Swvl was targeting to make the fare prices constant and have timely rides.

Although the service was launched during the nation locked down, Kenyans were eager to try the service. Especially those who had enjoyed their short distance ride sharing services. The long distance ride sharing business did not catch up. The Matatu operators would switch from the SWVL service when there was high demand. They needed the flexibility to decided and set the fare prices on their own.

From the experience of SWVL, we can learn how the market operates and what needs should be addressed. The Kenyan market is not ready for another `uber' like product for long distance ride sharing, but it needs a ridesharing solution that empowers both the driver and the passengers.

\subsection{Carpool World}

Carpool World is a ridesharing system developed by a French Company, Planète Covoiturage inc, \citep{carpoolworld}. It offers its services all over the world, connecting drivers and passengers. It has been in operation since 2000. It has listings for both drivers and passengers, offering carpool and vanpool. The majority of its users are from developing countries Singapore, Philippines, and India.

Very few Kenyans use this system. Those who use it only make trips within Nairobi. Probably because it is not developed to adopt the local needs. For example a Kenyan would like to search and see some of the local towns like Eldoret, Kisii, Naivasha.

\subsection{Uniqueness of the Suggested Solution}

\begin{enumerate}
    \item Empower both drivers and passengers

          Enable drivers decide the amount passengers will pay to cost share.
          Passengers to also have a way to choose the driver they need among those offering services.

    \item  Provide a system that is customized to the Kenyans

          Provide Local routes, and local relevant information regarding transportation
\end{enumerate}

