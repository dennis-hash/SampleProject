% Abstract
This project aims to research and implement a distributed long distance ridesharing system that connects drivers and passengers travelling to the same destination locations. Ridesharing is an interesting solution to some social problems like energy consumption, road congestion, providing quality travelling services and others. \citep{noland2006}.

Ridesharing systems have been widely implemented and used in US and Europe since WW II\@. There are a lot of lessons that the Kenyan market can learn from them. There has been a lot of innovation caused by `ride sharing', the like of Uber and lyft, which follow a relatively different business model to ridesharing which is based on helping drivers connect with passengers.

There is a gab for a social entrepreneurs to offer quality commuting services, using ridesharing technology. Some of the existing players in this sector have extended their services from short distance rides to long distance, due to the demand for such services \citep{swvlalvinwanjala}