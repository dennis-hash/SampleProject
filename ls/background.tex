% Background
Ridesharing began in the US during the World War II,\citep{ferguson1997}. The government encouraged carsharing to save rubber and fuel resources to be used in the war effort. Workers were incouraged to use the same car to and from work. During this time works in factories used notice boards to connect drivers and passengers.

After the war ridesharing services declined. The services later emerged during the 1970s due to the oil crisis. During this time corporates established internet notice boards and telephone-based computerized ridematching. They saw this as an opportunity to cut down fuel consumption and reduce their operation costs.

In the recent years there has been an increased interest in ridesharing services. These services are built on internet and GPS-smartphones. The services have transformed the industry. They have put innovation into the transit services.

It is estimated that in the next decade there will be a greater intergration of services, technology and policy support for ridesharing,\citep{chan2012}. This is due to concerns for energy, congestion, climate change and dependency on oil.