% Introduction

This project aims to research and implement a distributed long distance ridesharing system that connects drivers and passengers travelling to the same destination locations. Ridesharing is an interesting solution to some social problems like energy consumption, road congestion, providing quality travelling services and others. \citep{noland2006}.

Ridesharing systems have been widely implemented and used in US and Europe since WW II\@. There are a lot of lessons that the Kenyan market can learn from them. There has been a lot of innovation caused by `ride sharing', the like of Uber and lyft, which follow a relatively different business model to ridesharing which is based on helping drivers connect with passengers.

There is a gap for a social entrepreneurs to offer quality commuting services, using ridesharing technology. Some of the existing players in this sector have extended their services from short distance rides to long distance, due to the demand for such services \citep{swvlalvinwanjala}

Ridesharing is also known as liftsharing or car sharing in the UK\@. This is different from the terms `carsharing' in North America or `car clubs' in the UK, which refer to short term auto use of a car from a fleet of cars, that are shared hourly by passengers, \citep{shaheen2009}.

Ridesharing is the sharing of a cars journey, so that one person drives, preventing the need for the other people to drive themselves to the location. The driver and the passenger are travelling towards the same direction or from the same starting point, \citep{chan2012}. When payment is involved it is not for profitable reasons but to enable to cover the cost and services for the journey.

There has been a lot of interest in the ridesharing services in the recent years. This is because of the use of technology and easy access to internet services. Many people would prefer to travel on a private car than the public vehicles.

Ridesharing is seen as a solution to reducing congestion,  offering quality services to people, and reducing energy consumption \citep{noland2006}. Governments have put in place policies to encourage ridesharing services.